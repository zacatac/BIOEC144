\documentclass[11pt]{article}
\usepackage{amsmath,textcomp,amssymb,geometry,graphicx,tikz,cancel}
\usepackage{algpseudocode,algorithm}
\usepackage[colorlinks=true,urlcolor=blue]{hyperref}
\usepackage[T1]{fontenc}

\def\Name{Zackery Field}  % Your name
\def\Sec{vargas 002}  % Your GSI's name and discussion section
%\def\Login{} % Your login
\def\Homework{1}%Number of Homework
\def\Session{Spring 2014}

\title{Homework 1: Foundational Concepts and Personal Interest}
\author{\Name, Section: \Sec}%, \texttt{\Login}}
\markboth{CS170 --\Session\  Homework \Homework\ \Name, section \Sec}
{CS170--\Session\ Homework \Homework\ \Name, section \Sec}%, \texttt{\Login}}
\pagestyle{myheadings}

\begin{document}
\maketitle

\section*{1. Data Science}

Data in and of itself does not provide any insight or knowledge. Instead, 
what is often sought is the information within a dataset. The process of extracting
this information is not always straightforward. If we have a huge mass of text (the data)
and we seek to find all of the phone numbers within that text (the information) it
is fairly easy to use a tool like grep to extract that information. What if instead the
dataset is a genome, and the information that we seek to extract is the location of 
the protein coding sequences in that genome? Data scienctists seek to extract information
from data; always trying to make their process in keeping with the scientific method.

The general public often confuses data science with big data. The top few hits of a
\href{https://www.google.com/search?q=data+science+npr&oq=data+science+npr&aqs=chrome..69i57j0.3516j0j7&sourceid=chrome&espv=210&es_sm=91&ie=UTF-8}{google search} 
of ``data science npr'' are actually links to articles on big data, not data science. In business,
data science often refers to a method of extracting information that is useful, but
often not scientific. A more apt term for the 'data science' that businesses utilize
is \href{http://en.wikipedia.org/wiki/Analytics}{analytics}.


\section*{2.  Computational biology and bioinformatics}

My interpreation of the 
\href{http://en.wikipedia.org/wiki/Bioinformatics}{bioinformatics} 
wiki page is that bioinformatics is a subfield of data science. 
In particular, bioinformatics seeks to identify and classify the information
contained within data collected from biological systems. 
Additionally, it is a subfield of database science as bioinformaticists are also
concerned with the storage, retrieval, and organization of biological. 

On the other hand, computational biology is concerned with the computational 
modeling of biological systems in order to develop some new insight into 
how these systems function. 
This \href{http://rbaltman.wordpress.com/2009/02/18/bioinformatics-computational-biology-same-no/}{blog post}
does a good job of differentating between these two fields, both involving computation 
and biology. A biologist would participate in the field of computational biology, using
tools provided to them by engineers to uncover new knowledge. An engineer would 
create the tools that the biologist utilizes, this is the field of bioinformatics.


\section*{3. Ideal computational biology graduate program}

Given the above definition of computational biology as being a scientific field that
utilizes tools that come out of bioinformatics, the ideal program would not be solely
computational. The program would require the use of some computational tools however.
These tools would be necessary to extract some novel information about the 
biological system in question. Many courses in biology would be necessary in 
order to understand the complexities of the system being studied. 
In order to tweak the tools to meet the needs of a particular
project, some programming experience would be necessary. This programming experience
could come from a few CS courses, if the student has not already gained this experience elsewhere.
Since computational biology is a scientific field, there is no need for the student's project
to be practical. Any thesis topic that led to some new insight into a biological system,
and used computational tools to develop that insight would be acceptable. 


\section*{4. Lecture notes and commentary}

Computational
\begin{itemize}
\item[1.] K-means and hierarchical clustering
\item[2.] Supervised learning methods
\item[3.] Graph theory and network theory
\end{itemize}

Bioinformatics
\begin{itemize}
\item[1.] Genome annotation pipelines
\item[2.] Pathway databases and prediction
\item[3.] Protein functional site prediction
\end{itemize}

\end{document}